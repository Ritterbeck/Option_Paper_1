\documentclass[12pt]{article}

\usepackage{amsmath}
\usepackage{amssymb}
\usepackage{amsthm}

\theoremstyle{definition}
\newtheorem{definition}{Definition}

\theoremstyle{plain}
\newtheorem{corollarly}{Corollary}
\newtheorem{lemma}{Lemma}
\newtheorem{theorem}{Theorem}

\title{On the Generalized Black-Scholes-Merton Analytical Solution}
\author{Brent A. Ritterbeck \\ brent@ritterbeck.co}
\date{\today}

\begin{document}
\maketitle

\begin{abstract}
This is the abstract.
\end{abstract}

\section{Theory}

\begin{theorem} \label{eur_van_call_val}
The value $V_{c}$ of a European vanilla call option is given by
\begin{equation*}
V_{c} = Se^{\left ( b - r \right ) \left (T - t \right )} \Phi \left ( d_{1} \right )
    - Xe^{-r \left ( T - t \right )} \Phi \left ( d_{2} \right ),
\end{equation*}
where $d_{1}$ and $d_{2}$ are as defined in
\end{theorem}
\begin{proof}
\end{proof}

\begin{theorem} \label{eur_van_put_val}
The value $V_{p}$ of a European vanilla put option is given by
\begin{equation*}
V_{p} = Xe^{-r \left ( T - t \right )} \Phi \left ( -d_{2} \right )
    - Se^{\left ( b - r \right ) \left (T - t \right )} \Phi \left ( -d_{1} \right ),
\end{equation*}
where $d_{1}$ and $d_{2}$ are as defined in
\end{theorem}
\begin{proof}
\end{proof}

\begin{theorem} \label{eur_van_call_del}
The delta $\Delta_{c}$ of a European vanilla call option is given by
\begin{equation*}
\Delta_{c} = e^{\left ( b - r \right ) \left ( T - t \right )} \Phi \left ( d_{1} \right ),
\end{equation*}
where $d_{1}$ and $d_{2}$ are as defined in
\end{theorem}
\begin{proof}
\end{proof}

\begin{theorem} \label{eur_van_put_del}
The delta $\Delta_{p}$ of a European vanilla put option is given by
\begin{equation*}
\Delta_{p} = -e^{\left ( b - r \right ) \left ( T - t \right )} \Phi \left ( -d_{1} \right ),
\end{equation*}
where $d_{1}$ and $d_{2}$ are as defined in
\end{theorem}
\begin{proof}
\end{proof}

\begin{theorem} \label{eur_van_call_gam}
The gamma $\Gamma_{c}$ of a European vanilla call option is given by
\begin{equation*}
\Gamma_{c} =
\end{equation*}
where $d_{1}$ and $d_{2}$ are as defined in
\end{theorem}
\begin{proof}
\end{proof}

\begin{theorem} \label{eur_van_put_gam}
The gamma $\Gamma_{p}$ of a European vanilla put option is given by
\begin{equation*}
\Gamma_{p} =
\end{equation*}
where $d_{1}$ and $d_{2}$ are as defined in
\end{theorem}
\begin{proof}
\end{proof}

\begin{theorem} \label{eur_van_call_rho}
The rho $\rho_{c}$ of a European vanilla call option is given by
\begin{equation*}
\rho_{c} =
\end{equation*}
where $d_{1}$ and $d_{2}$ are as defined in
\end{theorem}
\begin{proof}
\end{proof}

\begin{theorem} \label{eur_van_put_rho}
The rho $\rho_{p}$ of a European vanilla put option is given by
\begin{equation*}
\rho_{p} = 
\end{equation*}
where $d_{1}$ and $d_{2}$ are as defined in
\end{theorem}
\begin{proof}
\end{proof}

\begin{theorem} \label{eur_van_call_the}
The theta $\Theta_{c}$ of a European vanilla call option is given by
\begin{equation*}
\Theta_{c} = 
\end{equation*}
where $d_{1}$ and $d_{2}$ are as defined in
\end{theorem}
\begin{proof}
\end{proof}

\begin{theorem} \label{eur_van_put_the}
The theta $\Theta_{p}$ of a European vanilla put option is given by
\begin{equation*}
\Theta_{p} =
\end{equation*}
where $d_{1}$ and $d_{2}$ are as defined in
\end{theorem}
\begin{proof}
\end{proof}

\begin{theorem} \label{eur_van_call_veg}
The vega $\Lambda_{c}$ of a European vanilla call option is given by
\begin{equation*}
\Lambda_{c} =
\end{equation*}
where $d_{1}$ and $d_{2}$ are as defined in
\end{theorem}
\begin{proof}
\end{proof}

\begin{theorem} \label{eur_van_put_veg}
The vega $\Lambda_{p}$ of a European vanilla put option is given by
\begin{equation*}
\Lambda_{p} = 
\end{equation*}
where $d_{1}$ and $d_{2}$ are as defined in
\end{theorem}
\begin{proof}
\end{proof}

\section{Implementation}

As this paper is meant to be pedagogical in nature, we will implement the
previously discussed material in non-object oriented Python.

\section{Recommended Texts}
See \cite{HAUG2007} and \cite{CHIN2017} for more information.

\bibliographystyle{alpha}
\bibliography{paper}

\end{document}
